% Nejprve uvedeme tridu dokumentu s volbami
\documentclass[slovak,master,dept460,male,cpp,cpdeclaration]{diploma}
% Dalsi doplnujici baliky maker
\usepackage[autostyle=true,czech=quotes]{csquotes} % korektni sazba uvozovek, podpora pro balik biblatex
%\usepackage[backend=biber, style=iso-numeric, alldates=iso]{biblatex} % bibliografie
\usepackage{dcolumn} % sloupce tabulky s ciselnymi hodnotami
\usepackage{subfig} % makra pro "podobrazky" a "podtabulky"
\usepackage{verbatim}
\usepackage{cite}
\nocite{*}
% Zadame pozadovane vstupy pro generovani titulnich stran.
\ThesisAuthor{Michal Falát}

\CzechThesisTitle{Analýza řidiče za pomocí sférických kamer}

\EnglishThesisTitle{Driver Analysis Using Spherical Cameras}

\SubmissionDate{1. apríla 2020}

% Pokud nechceme nikomu dekovat makro zapoznamkujeme.
\Thanks{Rád by som poďakoval môjmu vedúcemu práce Ing. Radovanovi Fusekovi za pomoc a ochotu pri vypracovaní diplomovej práce}

% Zadame cestu a jmeno souboru ci nekolika souboru s digitalizovanou podobou zadani prace.
% Pokud toto makro zapoznamkujeme sazi se stranka s upozornenim.
\ThesisAssignmentImagePath{Figures/Assignment1.jpg}
% \SlovakBachelorMaleAuthorDeclaration

% Zadame soubor s digitalizovanou podobou prohlaseni autora zaverecne prace.
% Pokud toto makro zapoznamkujeme sazi se cisty text prohlaseni.ss
% \AuthorDeclarationImageFile{Figures/AuthorDeclaration.jpg}


% Zadame soubor s digitalizovanou podobou souhlasu spolupracujici prav. nebo fyz. osoby.
% Pokud toto makro zapoznamkujeme sazi se cisty text souhlasu.
% \CooperatingPersonsDeclarationImageFile{Figures/CoopPersonDeclaration.jpg}

\CzechAbstract{Hlavnou témou diplomovej práce je rozpoznávanie a analýza vodiča v aute pomocou sférických kamier. Táto práca je rozdelená do viacerých samostatných častí. Prvá časť spočíva v samotnej detekcii ľudí a ich aktivít sférickou kamerou, hľadanie nedostatkov a nájdenie optimálnych parametrov pre čo najefektívnejšiu detekciu. Druhá časť je zameraná na porovnanie jednotlivych knižníc a metód, ktoré sa používaju na analýzu ľudského tela a tváre v obraze. Posledná časť je venovaná porovnaniu týchto metód s použitím reálnych dát zachytených sférickou kamerou a zhrnutie výsledkov. }

\CzechKeywords{Sférická kamera, detekcia obrazu, analýza ľudskej tváre, detekcia ľudí, vodič}

\EnglishAbstract{Main focus of this Diploma thesis is detection and analysis of driver in car with help of spherical cameras. This thesis is divided into few parts. The first part is about detection itself, detection of people by spherical cameras, research of disadvantages and finding optimal parameters for most efficnet detection. The second aprt is focused on comparision of libraries used for  human body and face detections. The last part is  about comparision of libraries with  real datas captured by  spherical camera and summary of results.  }

\EnglishKeywords{Spherical camera, image detection, analysis of human face, pedestrian detection, driver }

\AddAcronym{CPU}{Central processing unit}
\AddAcronym{FPS}{Frames per second}
\AddAcronym{GPU}{Graphical processing unit}
\AddAcronym{HOG}{Histogram oriented gradients}
\AddAcronym{OpenCV} {Open Source Computer Vision}
\AddAcronym{PX}{Pixel}



% Novy druh tabulkoveho sloupce, ve kterem jsou cisla zarovnana podle desetinne carkyss
\newcolumntype{d}[1]{D{,}{,}{#1}}


% Zacatek dokumentu
\begin{document}

% Nechame vysazet titulni strany.
\MakeTitlePages

% A nasleduje text zaverecne prace.
\section{Úvod}
\label{sec:Introduction}
V dnešnom modernom svete sú autá takmer každodennou súčasťou života ľudí. Mnohokrát sa ani nezamýšľame nad ich bezpečnosťou, ktorá je v prípade zrážky kľúčová. V súčasnosti nám pri jazde autom asistuje veľké množstvo systémov, ktoré zvyšujú bezpečnosť posádky, ale aj ostantých účastníkov cestnej premávky. Aj keď tieto systémy ešte stále nedokážu vodiča úplne nahradiť, dokážu mu výrazným spôsobom pomôcť napríklad v krízových situáciach. Výhodou takýchto systémov je ich rýchlejší reakčný čas oproti človeku. Takéto systémy spočívajú v použití rôznych snímačov alebo kamier, ktoré aktívne sledujú okolie ale aj interiér vozidla. Vďaka takýmto moderným technickým riešeniam je možné predísť rôznym  častokrát aj smrteľným dopravným nehodám. Výrobcovia áut sa čoraz častejšie snažia svoje systémy vylepšovať na čo najvyššiu možnú úroveň a poskytnúť tak vysoký level ochrany.\par Táto diplomová práca sa zameriava hlavne na problematiku analýzy vodiča pomocou detekcie obrazu zo sférickej (360-stupňovej) kamery. V diplomovej práci som sa venoval analýze videa z kamery umiestnenej v interiéri vozidla. Vhodným umiestnením kamery je možné získať obraz zpred auta, ale aj obraz vodiča sediaceho za volantom. V tejto práci som sa zameriaval na analýzu a spracovanie videa z interiéru vozidla na zachytenie ľudských aktivít vodiča. Aby som získal čo najväčšiu časť tela vodiča, je potrebné mať dostatočne veľký uhol záberu. Bežné kamery majú uhol záberu veľmi nízky, aby dokázal z malej vzdialenosti zachytiť celý snímaný objekt. Takýto problém sa naskytuje najpríklad aj v interiéri vozidla, kde je vzdialenoť kamery od snímaného objektu menej ako 1 meter, čo nemusí byť dostatočné na zosnímanie tela celého vodiča. Práve v takejto situácii je vhodné použiť širokouhlú prípadne sférickú kameru. Počas práce som mal k dispozicii viaceré kamery, s ktorými som zhotovol niekoľko desiatok videí v rôznych situáciach. Z takýchto videi som dokázal analyzovať a zistiť mnoho užitočných informácii, ktore sú spracované v tejto diplomovej práci. Tieto informácie som zbieral nahrávaním videa sférickymi kamerami za rôznych svetelnych  podmienok a pozicií vodiča. V tejto práci sú taktiež spomenuté problémy takejto analýzy, riešenia vzniknutých problémov, ale aj zhrnutie celkovej problematiky sledovania vodiča vo vozidle. V práci sú tiež zhrnuté ďalšie možnosti vylepšenia detekcie a porovnanie oproti klasickým kamerám.\par V nasledujúcich kapitolách je postupne rozobratá problematika snímania ľudských postáv v obrazoch, a skúmanie ich aktivít. Pre snímanie postavy som sa rozhodol použiť viacero metód, ktoré som následne porovnal a zanalyzoval. Aby som vedel vyhodnotiť správnu pozíciu vodiča, rozhodo lso msa  použiť neurónovú sieť, ktorú som trénoval na vlastnom datasete.\par V súčasnosti som taktiež nenašiel veľa riešení na spracovanie videa zo sférickej kamery a preto by som sa snažil zamerať túto prácu hlavne na túto oblasť. Pri analýze vodiča som taktiež nenašiel vhodné datasety z interiéru vozidla snímané sférickou kamerou.




\newpage
\section{Detekcia a analýza ľudského tela v obrazoch}
\label{sec:human body decection}

História detekcie postáv v obrazoch siaha až do polovice 20. storočia.  Mnoho ľudí už videlo obrovský potenciál detekcie napríklad v oblastiach medicíny , priemyslu, dopravy a mnoho ďalších. S nárastom technických možností postupne rástla aj motivácia využiť detekciu obrazu. Jeden z prvých vedeckých článkov v oblasti spracávaania obrazu \cite{rosenfeld1969}


\newpage
\subsection{Detekcia tváre}
Haar


\newpage
\subsection{HOG}
Obrazy obsahují různé objekty, které lze zařadit do jednotlivých tříd (jablko, obličej, chodec,
automobil, znaky abecedy atd.). Účelem příznakového rozpoznávání je extrahovat příznaky pro
popis těchto objektů, které vykazují určité hodnoty. Hodnoty těchto objektů se musí natolik lišit,
aby klasifikátor dokázal správně rozlišit, do které třídy objekt patří. V této práci pro extrakci
příznaků a jejích klasifikaci je použito několik metod.
2.1 Histogramy orientovaných gradientů
Příznakovou metodu HOG vytvořili Navneed Dalal a Bill Triggs v [1], kteří vyvinuli a otestovali
několik variant HOG deskriptorů. Ve své práci vyzkoušeli různé normalizační metody,
různé druhy gradientních operátorů a ukázali jak správně nastavit další parametry této detekční
metody v aplikaci detekování chodců.
Hlavní myšlenka metody HOG je, že objekt v obraze může být pomocí vzhledu a tvaru
charakterizován směrem hran, nebo intenzitou gradientů. Obraz se rozdělí na malé prostorové
oblasti tzv. buňky, a pro každou buňku se sestaví histogram orientovaných gradientů, který
je vypočítán ze všech pixelů z buňky. Je vhodné obraz před započetím výpočtů normalizovat,
například kontrastní normalizací, nebo normalizací osvětlení. Toho lze docílit shromažďováním
informací do histogramu nejen z jedné konkrétní buňky, ale z větší oblasti okolních buněk. Těchto
několik buněk dá dohromady tzv. blok.

\begin{equation}
\left(\sum_{n=1}^{\infty}a_{n}b_{n}\right)^{2} \leq
\sum_{n=1}^{\infty}a_{n}^{2} \cdot \sum_{n=1}^{\infty}b_{n}^{2}
\label{eq:A}
\end{equation}
OBRAZOK

\newpage
\section{Detekcia pozície vodiča}
\label{sec:Pose detection}
Nieco  o detekciach pozicie


\newpage
\subsection{OpenPOSE}
openPose


\newpage
\subsection{TensorFlow}
tensorflow


\newpage
\subsection{Ostatné metódy}
WrnchAI





\newpage
\section{Využitie sférických kamier v automobiloch}
\label{sec:Spherical cameras}
Počas vypracovania diplomovej práce som mal zapožičané 2 sférické kamery. 

\subsection{Technické parametre}
Go PRO:

THETA:

Senzor	FishEye CMOS 2x12MPix
Maximálne rozlíšenie (Video)	4K  30fps
Maximálne rozlíšenie (Fotografia)	14.5MP (5376x2688px)
Svetelnosť	f2.0
Vnútorná pamäť	19GB


\newpage
\subsection{Použitie v analýze videa}
Problem s formatom videa, rozlisenim , deformaciou  hroe a dole



\newpage
\section{Program}
\label{sec:Program}
Zhrnutie vysledkov


\newpage
\subsection{Požiadavky a návrh programu}
Architekruta , schemy


\newpage
\subsection{Pozícia vodiča}


\newpage
\subsubsection{Detekcia}
NN klasifikator


\newpage
\subsubsection{Neurónová sieť}
NN klasifikator



\newpage
\subsection{Orientácia hlavy}


\newpage
\subsection{Výstup programu}
obrazky

\newpage
\subsection{Porovnanie výsledkov}
porovnanie


\newpage
\subsection{Využitie zozbieraných dát}
porovnanie


\newpage
\subsection{Používateľská príručka}
python program.py --use-openPose=true


\newpage
\section{Možnosti vylepšenia detekcie}
\label{sec:Možnosti vylepšenia detekcie}
Zhrnutie vysledkov


\newpage
\section{Záver}
\label{sec:Zaver}
Zhrnutie vysledkov







\bibliography{literatura}
\bibliographystyle{plain}














\end{document}
